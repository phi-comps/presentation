\documentclass{beamer}

\mode<presentation>

\usetheme{Frankfurt}

\setbeamertemplate{navigation symbols}{}
% \setbeamertemplate{frametitle continuation}[from second][(contd.)]
\setbeamertemplate{frametitle continuation}[from second][]

\title{Some Title}
\subtitle{And Maybe a Subtitle}
\author[Illig \and Orvis \and Segawa \and Spinale]{Peter Illig \and Eli Orvis \and Yukihiko Segawa \and Nick Spinale}
\institute[Carleton]{Carleton College}
\date{February 20\textsuperscript{th}, 2018}

\usepackage[english]{babel}
\usepackage[utf8x]{inputenc}
\usepackage[T1]{fontenc}
\usepackage{amsthm,amssymb,amsmath}
\usepackage{graphicx}

\newcommand{\N}{\mathbb{N}}
\newcommand{\Z}{\mathbb{Z}}
\newcommand{\Q}{\mathbb{Q}}
\newcommand{\CC}{\mathbb{C}}
\newcommand{\g}{\gamma}
\newcommand{\xg}{x-\gamma}
\newcommand{\pxg}{(x-\gamma)}

\begin{document}

\newtheorem{thm}{Theorem}[section]
\newtheorem{prop}[thm]{Proposition}
\newtheorem{cor}[thm]{Corollary}
\newtheorem{obs}[thm]{Observation}
\newtheorem{defn}[thm]{Definition}
\newtheorem{exmp}[thm]{Example}
\newtheorem{remk}[thm]{Remark}

\maketitle

\begin{frame}
  \frametitle{Contents}
  \tableofcontents
\end{frame}

\section{Introduction}

\begin{frame}
  \frametitle{A Title}
  Contents of the slide
\end{frame}


\section{Simplifying the Problem}

\begin{frame}[allowframebreaks]{Roots of $f^n(x)$}

  % TODO show root tree?

  $f(x) = (x-\gamma)^2+\gamma+m$

  \begin{itemize}[\null]
    \item The roots of $f(x)$ are $\gamma\pm\sqrt{-m-\gamma}$
    \item If $\alpha$ is a root of $f^n(x)$, then $\gamma\pm\sqrt{\alpha-m-\gamma}$ are roots of $f^{n+1}(x)$
  \end{itemize}

  \begin{obs}
    For $n>0$, the roots of $f^n(x)$ are, with $n$ radicals:
    $$\gamma\pm\sqrt{-m\pm\sqrt{-m\pm\sqrt{-m\pm\ldots\sqrt{-m-\gamma}}}}$$
  \end{obs}

  \framebreak

  \begin{obs}
    For $n>0$, the roots of $f^n(x)$ are, with $n$ radicals:
    $$\gamma\pm\sqrt{-m\pm\sqrt{-m\pm\sqrt{-m\pm\ldots\sqrt{-m-\gamma}}}}$$
  \end{obs}

  For notational convenience, define $\beta : \Sigma^* \rightarrow \CC$ where
  \begin{align*}
    \beta_{\epsilon} &= -\gamma \\
    \beta_{0s} &= \sqrt{-m+\beta_s} \\ 
    \beta_{1s} &= -\sqrt{-m+\beta_s}
  \end{align*}

  For $n>0$, the roots of $f^n(x)$ are exactly $\{\;\gamma+\beta_s\mid s\in\Sigma^n\;\}$.

\end{frame}



\end{document}




