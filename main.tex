\documentclass{beamer}

\mode<presentation>

\usetheme{Frankfurt}

\setbeamertemplate{navigation symbols}{}
% \setbeamertemplate{frametitle continuation}[from second][(contd.)]
\setbeamertemplate{frametitle continuation}[from second][]

\title{Some Title}
\subtitle{And Maybe a Subtitle}
\author[Illig \and Orvis \and Segawa \and Spinale]{Peter Illig \and Eli Orvis \and Yukihiko Segawa \and Nick Spinale}
\institute[Carleton]{Carleton College}
\date{February 20\textsuperscript{th}, 2018}

\usepackage[english]{babel}
\usepackage[utf8x]{inputenc}
\usepackage[T1]{fontenc}
\usepackage{amsthm,amssymb,amsmath}
\usepackage{graphicx}
\usepackage{tikz}
\usepackage[normalem]{ulem}

\newcommand{\hcancel}[1]{ % red strikethrough
    \tikz[baseline=(tocancel.base)]{
        \node[inner sep=0pt,outer sep=0pt] (tocancel) {#1};
        \draw[red] (tocancel.south west) -- (tocancel.north east);}}
\newcommand{\hunder}[1]{ % red underline
    \tikz[baseline=(tocancel.base)]{
        \node[inner sep=0pt,outer sep=0pt] (tocancel) {#1};
        \draw[red] (tocancel.south west) -- (tocancel.south east);}}

\newcommand{\N}{\mathbb{N}}
\newcommand{\Z}{\mathbb{Z}}
\newcommand{\Q}{\mathbb{Q}}
\newcommand{\CC}{\mathbb{C}}
\newcommand{\g}{\gamma}
\newcommand{\xg}{x-\gamma}
\newcommand{\pxg}{(x-\gamma)}

\begin{document}

\newtheorem{thm}{Theorem}[section]
\newtheorem{prop}[thm]{Proposition}
\newtheorem{cor}[thm]{Corollary}
\newtheorem{obs}[thm]{Observation}
\newtheorem{defn}[thm]{Definition}
\newtheorem{exmp}[thm]{Example}
\newtheorem{remk}[thm]{Remark}

\maketitle

\begin{frame}
  \frametitle{Contents}
  \tableofcontents
\end{frame}

\section{Introduction}

\begin{frame}
  \frametitle{A Title}
  Contents of the slide
\end{frame}


\section{Simplifying the Problem}

\begin{frame}[allowframebreaks]{Roots of $f^n(x)$}

  % TODO show root tree?

  $f(x) = (x-\gamma)^2+\gamma+m$

  \begin{itemize}[\null]
    \item The roots of $f(x)$ are $\gamma\pm\sqrt{-m-\gamma}$
    \item If $\alpha$ is a root of $f^n(x)$, then $\gamma\pm\sqrt{\alpha-m-\gamma}$ are roots of $f^{n+1}(x)$
  \end{itemize}

  \begin{obs}
    For $n>0$, the roots of $f^n(x)$ are, with $n$ radicals:
    $$\gamma\pm\sqrt{-m\pm\sqrt{-m\pm\sqrt{-m\pm\ldots\sqrt{-m-\gamma}}}}$$
  \end{obs}

  \framebreak

  \begin{obs}
    For $n>0$, the roots of $f^n(x)$ are, with $n$ radicals:
    $$\gamma\pm\sqrt{-m\pm\sqrt{-m\pm\sqrt{-m\pm\ldots\sqrt{-m-\gamma}}}}$$
  \end{obs}

  For notational convenience, define $\beta : \Sigma^* \rightarrow \CC$ where
  \begin{align*}
    \beta_{\epsilon} &= -\gamma \\
    \beta_{0s} &= \sqrt{-m+\beta_s} \\ 
    \beta_{1s} &= -\sqrt{-m+\beta_s}
  \end{align*}

  For $n>0$, the roots of $f^n(x)$ are exactly $\{\;\gamma+\beta_s\mid s\in\Sigma^n\;\}$.

\end{frame}


\section{Newly Reducible Third Iterates, Part 1}

\begin{frame}[allowframebreaks]{Newly Reducible Third Iterates Part 1}


Let

  \begin{itemize}
    \item $p_1(x) = a + b (\xg) + c (\xg)^{2} + d(\xg)^{3} + (\xg)^{4}$
    \item $p_2(x) = a - b (\xg) + c (\xg)^{2} - d (\xg)^{3} + (\xg)^{4}$
  \end{itemize}

If $f^3 = p_1(p_2)$, then

\begin{align*}
\gamma + m^{4} + 2 m^{3} + m^{2} + m &= a^2 \\
4 m^{3} + 4 m^{2} &= 2ac - b^2 \\
6 m^{2} + 2 m &= 2a - 2bd + c^2 \\
4 m &= 2c - d^{2}
\end{align*}

\framebreak

Every newly reducible third iterate is a rational point on this surface!

\end{frame}

\begin{frame}
\frametitle{Simplifying the System}

We can simplify this system of equations using linear substitutions and the quadratic formula to get

\begin{align*}
\gamma= & \pm\beta \left(- \frac{3 d^{6}}{16} - \frac{d^{4} m}{2} - \frac{d^{2} m^{2}}{2} - \frac{d^{2} m}{2}\right) \\
& + \frac{17 d^{8}}{64} + \frac{5 m}{4} d^{6} + \frac{11 d^{4}}{4} m^{2} + \frac{7 m}{4} d^{4} + 2 d^{2} m^{3} + 2 d^{2} m^{2} - m
\end{align*}

\pause

Note that every expression in this formula is a rational function of $d$, $m$, and $\beta$

\end{frame}

\begin{frame}
\frametitle{All about $\beta$}

Let's look at $\beta$

\begin{equation*}
\beta = \sqrt{2 d^{4} + 8 d^{2} m + 16 m^{2} + 16 m}
\end{equation*}

\pause 

Letting $\beta = y$ and $d = s$ we have the surface

\begin{equation*}
S:\ y^2 = 2 s^4 + 8 m s^2 + 16 m^2 + 16 m
\end{equation*}

\end{frame}

\begin{frame}
\frametitle{The Curves $C_m$}

We want to explore the surface $S$, by considering the curve $C_m$ that results from a fixed value of $m$.

\pause

\begin{equation*}
C_{m_0}:\ y^2 = 2 s^4 + 8 s^2 (m_0) + 16 (m_0)^2 + 16 m_0
\end{equation*}

\pause

Some questions to consider:

\begin{itemize}
\item What are the roots of $C_m$?
\pause
\item What is the genus of $C_m$?
\pause
\item Does $C_m$ have rational points?  If so, how many?
\end{itemize}

\end{frame}

\begin{frame}
\frametitle{The roots of $C_m$}

What do the roots of $C_m$ tell us?

\pause

If $C_m$ has repeated roots, it's a conic!  Otherwise, it's an elliptic curve.

\pause

So when does $C_m$ have repeated roots?

\end{frame}

\begin{frame}
\frametitle{The roots of $C_m$}

\begin{align*}
C_{m_0}:\ y^2 & = 2 s^4 + 8 s^2 (m_0) + 16 (m_0)^2 + 16 m_0 \\
\pause
& = 2(s^2)^2 + 8 s^2 (m_0) + 16 (m_0)^2 + 16 m_0
\end{align*}

\pause

Using the quadratic formula we get that $C_m$ has a repeated root if and only $\sqrt{-2m-m^2} = 0$ or $2(-m\pm\sqrt{-2m-m^2}) = 0$.  This happens when $m \in \{0, -1, -2\}$.



\end{frame}

\end

\end{frame}

\section{Newly Reducible Third Iterates, Part 2}

\begin{frame}[allowframebreaks]
  \frametitle{A New Perspective}
  $$S:y^2 = 2s^4 + 8ms^2 + 16m^2 + 16m$$
  \phantom{hi}
  \includegraphics[width=\textwidth]{SPlot.png}
  
\framebreak
  
  So far,
	$$S:y^2 = 2s^4 + 8ms^2 + 16m^2 + 16m$$
	\phantom{hi}
	\includegraphics[width=\textwidth]{ECPlot.png}

\framebreak

	$$S:y^2 = 16m^2 + (16+8s^2)m + 2s^4$$
	\phantom{hi}
	\includegraphics[width=\textwidth]{CSPlot.png}
	
	\framebreak
	
	This is a conic! $$y^2 = am^2 + bm + c$$
	\phantom{hi}
	\includegraphics[width=\textwidth]{CSPlot.png}
\end{frame}

\begin{frame}
	\frametitle{Rational Projection}
	For any $s$, we have
	$$y^2 = 16m^2 + (16+8s^2)m + 2s^4.$$
	\pause
	\begin{center}
	\includegraphics[width=0.5\textwidth]{s0Plot.png} \\
	Example: $s=0$
	\end{center}
\end{frame}
	
\begin{frame}
	\frametitle{Rational Projection}
	$$S: y^2 = 16m^2 + (16+8s^2)m + 2s^4$$
	\pause
	\begin{obs}
		We're only looking for \textbf{rational} solutions.
	\end{obs}
	\pause
	$$\mbox{Let }y = \frac{Y}{Z}\mbox{ and }m = \frac{M}{Z}.$$
	\pause
	\begin{defn}
		The \textbf{homogeneous form} of $S$ is
		$$S: Y^2 = 16 M^2 + (8 s^2 + 16) M Z + 2 s^4 Z^2.$$
	\end{defn}
\end{frame}

\begin{frame}
	\frametitle{Rational Projection}
	If $Z=0$...
	\pause
	$$Y^2 = 16 M^2 + (8 s^2 + 16) M Z + 2 s^4 Z^2$$
	\pause
	$$Y^2 = 16 M^2 +\text{ \hcancel{\ensuremath{(8 s^2 + 16) M Z}}} + \text{ \hcancel{\ensuremath{2 s^4 Z^2}}}$$
	\pause
	$$ Y^2 = 16 M^2 $$
	\pause
	$$ Y=\pm 4 M$$
	\pause
	\begin{obs}
	The point $[M:Y:Z]=[1:4:0]$ is a solution to the homogeneous form of $S$.
	\end{obs}
\end{frame}

\begin{frame}
	\frametitle{Rational Projection}
	Geometrically, this is a line with slope $4$.
	\begin{center}
		\includegraphics[width=.6\textwidth]{s0AsymptotePlot.png} \\
		Example: $s=0$ and $y=4m+2$
	\end{center}
\end{frame}

\begin{frame}
	\frametitle{Rational Projection}
	To project from the point at infinity, take any line with slope 4.
	\begin{center}
		\includegraphics[width=.6\textwidth]{s0ProjectionPlot.png} \\
		Example: $s=0$ and $y=4m+r_0$
	\end{center}
\end{frame}

\begin{frame}
	\frametitle{Rational Projection}
	This intersects $S$ at a rational point:
	\begin{center}
		\includegraphics[width=.6\textwidth]{s0ProjectionIntersectionPlot.png} \\
		Example: $s=0$ and $y=4m+r_0$
	\end{center}
\end{frame}

\begin{frame}
	\frametitle{Rational Projection}
	Solving for this intersection point gives
	\begin{align*}
		m &= \frac{2 s^4-r_0^2}{8 r} \\
		\mbox{and }y &= \frac{-4 + r^2 - 4 s^2 + s^4}{2 r}
	\end{align*}
	$\mbox{where }r=\left(r_0-s^2-2\right).$\\
	\pause
	So for every rational $r$ and $s$, we get rational $m$ and $y$ such that
	$$ y^2 = 16m^2 + (16+8s^2)m + 2s^4 $$
\end{frame}

\begin{frame}
	\frametitle{Rational Projection}
	\begin{defn}
		We define this projection as $$ \phi(r,s) = (m(r,s),y(r,s)) $$ where $$m(r,s)=\frac{-4 - 4 r - r^2 - 4 s^2 - 2 r s^2 + s^4}{8r},$$ $$y(r,s)=\frac{-4 + r^2 - 4 s^2 + s^4}{2 r}  $$
	\end{defn}
\end{frame}

\begin{frame}
	\frametitle{Rational Projection}
	This gives us a value for $m$. Defining $f$ requires $m$ and $\gamma$. Luckily, we've already seen an equation for $\gamma$.
	\pause
	\begin{defn}
		\begin{align*}
			\begin{split}
				\gamma(r,s) = \pm\beta \left(- \frac{3 s^{6}}{16} - \frac{s^{4} m}{2} - \frac{s^{2} m^{2}}{2} - \frac{s^{2} m}{2}\right) + \frac{17 s^{8}}{64} + \frac{5 m}{4} s^{6} + \\ \frac{11 s^{4}}{4} m^{2} + \frac{7 m}{4} s^{4} + 2 s^{2} m^{3} + 2 s^{2} m^{2} - m
			\end{split}
		\end{align*}
	where $m=m(r,s)$ is given by our projection.
	\end{defn}
\end{frame}

\begin{frame}
	\frametitle{Rational Projection}
	\begin{example}
		If $r=1$ and $s=1$,
		$$\phi(r,s) = (m(r,s),y(r,s)) = \left(-\frac{7}{4},3\right)$$
		and
		$$\gamma(r,s) = \frac{1}{2}.$$
		\pause
		This gives the polynomial
		\begin{align*}
			f(x) &= \left(x-\frac{1}{2}\right)^2+\frac{1}{2}-\frac{7}{4} \\
			&= x^2 - x - 1.
		\end{align*}
		\pause
		This is the polynomial for the golden ratio!
	\end{example}
	
\end{frame}

\begin{frame}
	\frametitle{Rational Projection}
	\begin{itemize}
		\item<1-> Every newly reducible $f^3$ gives a point on $S$. \\
		\item<2-> So by choosing all $(r,s)$, we get all newly reducible $f^3$. \\
		\item<3-> However, we will also get some that are not \textbf{newly} reducible. \\
		\item<4-> How can we ensure that we get a newly reducible $f^3$?
	\end{itemize}
\end{frame}

\begin{frame}
	\frametitle{Finding Newly Reducible Third Iterates}
	Recall that
	\begin{align*}
		f\mbox{ is reducible } &\Leftrightarrow -m-\gamma\mbox{ is a square,} \\
		\mbox{and }f^2\mbox{ is newly reducible }&\Leftrightarrow2(-m\pm\sqrt{m^2+m+\gamma})\mbox{ is a square.}
	\end{align*}
	So if we have a point on $S$ and neither $-m-\gamma$ nor $m^2+m+\gamma$ is a square, $f^3$ is newly reducible.
\end{frame}

\begin{frame}
	\frametitle{Finding Newly Reducible Third Iterates}
	\begin{align*}
		\begin{split}
			-m-\gamma &=  \frac{1}{256 r^2} s^2 \left(r^2-2 (r+2) s^2+s^4-4\right)^2\left(4 + 2 r - s^2\right)
		\end{split}\\
		\begin{split}
			\begin{tabular}{r}$m^2+m+\gamma$ \\ \vphantom{x} \\ \vphantom{x} \end{tabular} & \begin{tabular}{l}$=$\\ \vphantom{x} \\ \vphantom{x} \end{tabular} \begin{tabular}{r} $\dfrac{1}{256 r^2}\left(r-s^2+2\right)^2 (16 + 16 r + 4 r^2 + 32 s^2 + 32 r s^2$ \\ $ + 4 r^2 s^2 - 2 r^3 s^2 + 8 s^4 + 12 r s^4 $ \\ $ + 5 r^2 s^4 - 8 s^6 - 4 r s^6 + s^8).$ \end{tabular}
		\end{split}
	\end{align*}
\end{frame}

\begin{frame}
	\frametitle{Finding Newly Reducible Third Iterates}
	\begin{align*}
		\begin{split}
			\left(4 + 2 r - s^2\right) \\
			\phantom{x} \\
			(16 + 16 r + 4 r^2 + 32 s^2 + 32 r s^2 + 4 r^2 s^2 - 2 r^3 s^2 + 8 s^4 + 12 r s^4 \\ + 5 r^2 s^4 - 8 s^6 - 4 r s^6 + s^8)
		\end{split}
	\end{align*}
\end{frame}

\begin{frame}
	\frametitle{Finding Newly Reducible Third Iterates}
	\begin{align*}
		\begin{split}
		\left(4 + 2 r - s^2\right) \\
		\phantom{x} \\
		(16 + 16 r + 4 r^2 + 32 s^2 + 32 r s^2 + 4 r^2 s^2 \hunder{$- 2 r^3 s^2$} + 8 s^4 + 12 r s^4 \\ + 5 r^2 s^4 - 8 s^6 - 4 r s^6 + s^8)
		\end{split}
	\end{align*}
	\pause
	Let $r$ be "big enough"
\end{frame}
























\end{document}




